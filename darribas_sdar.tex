\documentclass[12pt, a4paper]{article}

\usepackage{amssymb, amsmath, booktabs, dcolumn, graphics, graphicx, rotating, natbib, lscape, longtable, setspace, caption, makeidx, subfig, fullpage, marvosym, microtype, booktabs}
%\renewcommand{\baselinestretch}{1.5}
\usepackage{graphicx}
%\usepackage[style=authoryear, bibencoding=utf8, natbib=true, firstinits=true, maxcitenames=2, maxbibnames = 99, uniquename=false, backend=bibtex, backref=false,doi=false,isbn=false,url=false]{biblatex}
%\renewbibmacro{in:}{%
%  \ifentrytype{article}{}{%
%  \printtext{\bibstring{in}\intitlepunct}}}
%\bibliography{PhD.bib}
%\AtEveryBibitem{%
%  \clearfield{day}%
%  \clearfield{month}%
%  \clearfield{endday}%
%  \clearfield{endmonth}%
%}
\usepackage[pdftex,colorlinks=true,citecolor=red,urlcolor=magenta,pdfstartview=FitH]{hyperref}
      \pdfcompresslevel=9
      \hypersetup{
       		pdftitle={nl4SQ, Cultural Diversity and Popular Urban Amenities},
		   	pdfauthor={Jessie Bakens},
		   	pdfsubject={},
		   	pdfkeywords={}
      } 
\usepackage{color,hyperref}
\definecolor{darkblue}{rgb}{0.0,0.0,0.3}
\hypersetup{colorlinks,breaklinks,
            linkcolor=darkblue,urlcolor=darkblue,
            anchorcolor=darkblue,citecolor=darkblue}
          
\begin{document}
\renewcommand{\baselinestretch}{1}
\title{\bfseries Spatial Data, Analysis, and Regression - A mini course } 
\bigskip
\author{Daniel Arribas-Bel \bigskip 
\and 
{\small { \emph{School of Geography, Earth and Environmental Sciences,
University of Birmingham}}}  \\
{\small  {\emph{Birmingham, United Kingdom}}}  \\
\bigskip}
\date{}
%\date{\today}
\maketitle

\begin{abstract}
    
This resource contains the materials and structure suggested to run a mini
course on spatial data, analysis and regression of approximately 14 hours. The
course is structured along four lectures and four labs that require the use of
computers.

Lectures present an introductory overview of why it is important to explictly
consider space in quantitative analysis. The first session covers different
types of spatial data and motivates spatial analysis, introducing the concept
of spatial dependence and stressing its differences spatial heterogeneity. The
next session introduces spatial weights, the spatial lag operator and provides
an overview of the most basic tools of exploratory spatial data analysis
(ESDA). Third and fourth lectures delve into spatial regression. After a
motivation, time is spent on model specification, diagnostics and estimation,
concluding with an overview of software implementations of spatial econometric
techniques.

Computer labs provide practical lessons that solidify the concepts explained
in the lectures and allow the student to learn some of the main tools
available to carry out spatial analysis. The first session uses QGIS to open,
manipulate and transform spatial data. The second lab uses GeoDa as an
interactive tool to explore data and perform the main ESDA techniques. The
third lab covers the specification and estimation of spatial econometric
models using GeoDaSpace, while the fourth replicates its results using the
open-source Python library PySAL.

As a whole, this resource is intended for both instructors and students. The
latter can follow the structure of the sessions, get a sense of the main
topics through the slides provided and continue with the suggested readings.
The former can take it as an initial material and adapt it to their own
teaching practices, extending in areas they consider more relevant, or
skipping parts deemed inneccesary for their own needs. To that end, the course
is released as an open-source software project and licensed using
Creative-Commons, which allows reuse, remix and redistribution.

\bigskip

{\noindent\small 
\textbf{Keywords:} Spatial Analysis, course, open-source
}
\end{abstract}

\clearpage

\end{document}

